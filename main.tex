\documentclass[12pt]{article}
\usepackage[utf8]{inputenc}
\usepackage[numbers]{natbib}   % reference package
\usepackage{amsmath}          % american math society
\usepackage{amsfonts}         %--"-- fonts
\usepackage{amssymb}          %---"-- symbols
\usepackage{graphicx}         % needed for options for images
\usepackage{subcaption}       % using subcaptions
\usepackage{lmodern}  % for bold teletype font
\usepackage{xcolor}   % for \textcolor
\usepackage{listings}
\lstset{
  basicstyle=\ttfamily,
  columns=fullflexible,
  frame=single,
  breaklines=true,
  postbreak=\mbox{\textcolor{red}{$\hookrightarrow$}\space},
}


%% deal with geometry
\usepackage[
	top=1in,
	bottom=1in,
	left=1in,
	right=1in]{geometry}

%% set spacing, double for normal text, 1 1/2 for captions
\usepackage{setspace}
\doublespacing
\usepackage[font=onehalfspacing]{caption}

% bibliography stuff
\bibliographystyle{ieetran}

\title{Covariance Matrix Fit Users' Guide}
\author{Tom Carroll, Adam Lister and Brian Rebel}
\date{May 2019}

\begin{document}

\maketitle

\begin{abstract}
    This technote details a new method for performing fits for oscillation parameters by using covariance matrices.
\end{abstract}

\tableofcontents

\section{Introduction}

\section{Structure}

The CMF code lives within \lstinline{nova/CovarianceMatrixFit}. Within this directory there are several directories:

\begin{lstlisting}
core 		: the core classes that CMF is built on: EventLists, VarVals, ShifterAndWeighter
data		: contains root files for calibration systematic uncertainties
dataProducts: data products and structs commonly used in CMF analysis
fhicl		: contains all fhicl files
macros 		: useful .C files
modules 	: art modules and plugins which are run with fhicl files
scripts 	: scripts for, i.e. submitting to the grid
utilities   : only the bin utility lives here, this may be removed in the future
\end{lstlisting}

\section{Generating Covariance Matrices}

This section contains information on how to generate covariance matrices for a given systematic uncertainty both locally.

\subsection{Locally}

In order to generate a covariance matrix locally, a single fhicl file can be run \footnote{I'd recommend copying this to your \lstinline{/nova/ana/users/${USER}} area and running from there.},\\
\begin{lstlisting}[language=bash]
cmf_covariancematrixmakerjob.fcl
\end{lstlisting}
\footnote{Ensure you're using the CMF version of this file. Another version, \lstinline{covariancematrixmakerjob.fcl} exists, but is related to the FNEX framework and is not what you should be running.
}

Inside this fhicl file, there are three options that a user should configure:\\
\begin{lstlisting}
TREEFILE	: Path to EventList file. For now these are located in /nova/ana/users/brebel/skimmed
SYSTPAR		: Systematic parameter to vary
NITER		: Number of iterations of the systematic to run (i.e. number of universes)
\end{lstlisting}

The options for which systematics you can choose can be found in \lstinline{CMF_SystematicParameters.fcl}

Once these substitutions have been made, a covariance matrix can be generated with the following command

\begin{lstlisting}
art -c cmf_covariancematrixmakerjob.fcl
\end{lstlisting}

\subsection{On The Grid}

\end{document}
