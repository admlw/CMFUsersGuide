\section{Generating Covariance Matrices}

This section contains information on how to generate covariance matrices for a given systematic uncertainty both locally.

\subsection{Locally}

In order to generate a covariance matrix locally, a single fhicl file can be run \footnote{I'd recommend copying this to your \lstinline{/nova/ana/users/${USER}} area and running from there.},\\
\begin{lstlisting}[language=bash]
cmf_covariancematrixmakerjob.fcl
\end{lstlisting}
\footnote{Ensure you're using the CMF version of this file. Another version, \lstinline{covariancematrixmakerjob.fcl} exists, but is related to the FNEX framework and is not what you should be running.
}

Inside this fhicl file, there are three options that a user should configure:\\
\begin{lstlisting}
TREEFILE	: Path to EventList file. For now these are located in /nova/ana/users/brebel/skimmed
SYSTPAR		: Systematic parameter to vary
NUMITER		: Number of iterations of the systematic to run (i.e. number of universes)
\end{lstlisting}

The options for which systematics you can choose can be found in \lstinline{CMF_SystematicParameters.fcl}

Once these substitutions have been made, a covariance matrix can be generated with the following command

\begin{lstlisting}
art -c cmf_covariancematrixmakerjob.fcl
\end{lstlisting}

\subsection{On The Grid}

Running on the grid is made easy by the existence of a bash script, located in 
\begin{lstlisting}
CovarianceMatrixFit/scripts/CMF_Run_Covariance_Grid.sh
\end{lstlisting}
which has a usage:
\begin{lstlisting}
usage: CMF_Run_Covariance_Grid.sh <systematics> <eventlist file> <local products> <output dir>
<systematics>:    specify one of calib, genie, mec, nue, norm, reco, xsec1, xsec2, xsec3
<eventlist file>: full path to event list tree file in pnfs
<local products>: name of the local products directory
<output dir>:     top level directory for output matrix root files
\end{lstlisting}

where the \lstinline{<local products>} is the name of a tar file (without the \lstinline|.tar.bz2|) which should be stored in your scratch area (\lstinline|/pnfs/scratch/users/${USER}|).

This script by default will run a set of several uncertainties defined by the \lstinline|<systematics>| tag. Each systematic uncertainty will have by default 2000 systematic universes across 200 grid jobs. This can be modified by changing the variables in the \lstinline|setVariables()| function in the bash script.